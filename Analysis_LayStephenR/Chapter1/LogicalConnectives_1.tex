% !TEX TS-program = pdflatex
% !TEX encoding = UTF-8 Unicode

% This is a simple template for a LaTeX document using the "article" class.
% See "book", "report", "letter" for other types of document.

\documentclass[11pt]{article} % use larger type; default would be 10pt

\usepackage[utf8]{inputenc} % set input encoding (not needed with XeLaTeX)

%%% Examples of Article customizations
% These packages are optional, depending whether you want the features they provide.
% See the LaTeX Companion or other references for full information.

%%% PAGE DIMENSIONS
\usepackage{geometry} % to change the page dimensions
\geometry{a4paper} % or letterpaper (US) or a5paper or....
% \geometry{margin=2in} % for example, change the margins to 2 inches all round
% \geometry{landscape} % set up the page for landscape
%   read geometry.pdf for detailed page layout information

\usepackage{graphicx} % support the \includegraphics command and options

% \usepackage[parfill]{parskip} % Activate to begin paragraphs with an empty line rather than an indent

%%% PACKAGES
\usepackage{booktabs} % for much better looking tables
\usepackage{array} % for better arrays (eg matrices) in maths
\usepackage{paralist} % very flexible & customisable lists (eg. enumerate/enumerate, etc.)
\usepackage{verbatim} % adds environment for commenting out blocks of text & for better verbatim
\usepackage{subfig} % make it possible to include more than one captioned figure/table in a single float
\usepackage{amsmath}
\usepackage{multirow}
\usepackage{enumitem} \setenumerate[0]{label=(\alph*)}
% These packages are all incorporated in the memoir class to one degree or another...

%%% HEADERS & FOOTERS
\usepackage{fancyhdr} % This should be set AFTER setting up the page geometry
\pagestyle{fancy} % options: empty , plain , fancy
\renewcommand{\headrulewidth}{0pt} % customise the layout...
\lhead{}\chead{}\rhead{}
\lfoot{}\cfoot{\thepage}\rfoot{}

%%% SECTION TITLE APPEARANCE
\usepackage{sectsty}
\allsectionsfont{\normalfont\large\sffamily} % (See the fntguide.pdf for font help)
% (This matches ConTeXt defaults)

%%% ToC (table of contents) APPEARANCE
\usepackage[nottoc,notlof,notlot]{tocbibind} % Put the bibliography in the ToC
\usepackage[titles,subfigure]{tocloft} % Alter the style of the Table of Contents
\renewcommand{\cftsecfont}{\rmfamily\mdseries\upshape}
\renewcommand{\cftsecpagefont}{\rmfamily\mdseries\upshape} % No bold!

%%% END Article customizations

%%% The "real" document content comes below...

\title{Logical Connectives}
\author{Jon Beattie}
%\date{} % Activate to display a given date or no date (if empty),
         % otherwise the current date is printed 

\begin{document}
\maketitle

\section{Mark each statement as True or False. Justify each answer.}
\begin{enumerate}
	\item In order to be classified as a statement, a sentence must be true.
	\subitem False; statements must be either true or false, but not both.
	
	\item Some statements are both true and false.
	\subitem False; a statement cannot be both true and false.
	
	\item When statement $p$ is true, then its negation $\neg p$ is false.
	\subitem True; this is the definition of negation
	
	\item A statement and its negation may both be false.
	\subitem False; this violates the definition of negation. The negation of a statement must evaluate to the opposite truth value of the statement.
	
	\item In mathematical logic, the word ``or'' has an inclusive meaning.
	\subitem True; by default, ``or'' is inclusive. There is a seperate operation for ``exclusive or''.
\end{enumerate}

\section{Mark each statement as True or False. Justify each answer.}
\begin{enumerate}
	\item In an implication $p \implies q$, statement $p$ is referred to as the proposition.
	\subitem False; $p$ is referred to as the antecedent. The whole implication is a single proposition.
	
	\item The only case where $p \implies q$ is false is when $p$ is true and $q$ is false.
	\subitem True; this is the definition of implication.
	
	\item ``If $p$, then $q$ is equivalent to ``$p$ whenever $q$''.
	\subitem False; a statement that if false can imply a statement that is true. ``$p$ whenever $q$ is actually logical equivalence (\textit{if and only if}).
	
	\item The negation of a conjunction is the disjunction of the negation of the individual parts.
	\subitem True; The negation of a conjunction is a disjunction of negations. This is often referred to as DeMorgan's Law.
	
	\item The negation of $p \implies q$ is $q \implies p$.
	\subitem False; True will still imply true and false will still imply false. The negation of implication is ``$p$ and not $q$''.
	
\end{enumerate}

\section{Write the negation of each statement}
\begin{enumerate}
	\item$M$ is a cyclic subgroup.
	\subitem $M$ is not a cyclic subgroup.
	
	\item The interval [0,3] is finite.
	\subitem The interval [0,3] is not finite.
	
	\item The relation R is reflexive and symmetric.
	\subitem The relation R is neither reflexive or symmetric.
	
	\item The set $S$ is finite or denumerable.
	\subitem The set $S$ is infinite and uncountable.
	
	\item If $x>3$, then $f(x)>7$.
	\subitem $x>3$ and $f(x) \leq 7$.
	
	\item If $f$ is continuous and $A$ is connected, then $f(A)$ is connected.
	\subitem $f$ is continuous and $A$ is connected and $f(A)$ is not connected.
	
	\item If $K$ is compact, then $K$ is closed and bounded.
	\subitem $K$ is compact and $K$ is neither closed or bounded.
\end{enumerate}

\section{Write the negation of each statement}
\begin{enumerate}
	\item The relation R is transitive.
	\subitem The relation R is not transitive.
	
	\item The set of rational numbers is bounded.
	\subitem The set of rational numbers is unbounded.
	
	\item The function $f$ is injective and surjective.
	\subitem The function $f$ is neither injective or surjective.
	
	\item $x<5$ or $x>7$.
	\subitem $x \geq 5$ and $x \leq 7$.

	\item If $x$ is in $A$, then $f(x)$ is not in $B$.
	\subitem $x$ is in $A$ and $f(x)$ is in $B$.
	
	\item If $f$ is continuous, then $f(S)$ is closed and bounded.
	\subitem $f$ is continuous and $f(S)$ is open or unbounded.
	
	\item If $K$ is closed and bounded, then $K$ is compact.
	\subitem $K$ is closed and bounded and $K$ is not compact.

\end{enumerate}

\section{Identify the antecedent and the consequent in each statement.}
\begin{enumerate}
	\item $M$ has a zero eigenvalue only if $M$ is singular.
	\subitem Antecedent: $M$ has a zero eigenvalue.\subitem Consequent: $M$ is singular.
	
	\item Normality is a necessary condition for regularity.
	\subitem Antecedent: Regularity\subitem Consequent: Normality
	
	\item A sequence is bounded if it is Cauchy.
	\subitem Antecedent: A sequence is Cauchy.\subitem Consequent: It is bounded.
	
	\item If $x=5$, then $f(x)=14$.
	\subitem Antecedent: $x=5$\subitem Consequent: $f(x)=14$
\end{enumerate}

\section{Identify the antecedent and the consequent in each statement.}
\begin{enumerate}
	\item$5n$ is odd only if $n$ is odd.
	\subitem Antecedent: $5n$ is odd.\subitem Consequent: $n$ is odd.
	\item A sequence is convergent provided that it is monotone and bounded.
	\subitem Antecedent: A sequence is monotone and bounded.\subitem Consequent: It is convergent.
	\item A real sequence is Cauchy whenever it is convergent.
	\subitem Antecedent: A real sequence is convergent.\subitem Consequent: It is Cauchy
	\item Convergence is a sufficient condition for boundedness.
	\subitem Antecedent: A sequence is convergent.\subitem Consequent: It is bounded.
\end{enumerate}

\section{Construct a truth table for each statement.}
\begin{enumerate}
	\item $p \implies \neg q$
	\begin{center}
	\begin{tabular}{ |c|c|c|c| }
	\hline
	$p$ & $q$ & $\neg q$ & $p \implies \neg q$ \\
	\hline
	T & T & F & F \\
	T & F & T & T \\
	F & T & F & T \\
	F & F & T & T \\
	\hline
	\end{tabular}
	\end{center}
	\item $[p \land (p \implies q)] \implies q$
	\begin{center}
	\begin{tabular}{|c|c|c|c|c|}
	\hline
	$p$ & $q$ & $p \implies q$ & $p \land (p \implies q)$ & $[p \land (p \implies q)] \implies q$ \\
	\hline
	T & T & T & T & T \\
	T & F & F & F & T \\
	F & T & T & F & T \\
	F & F & T & F & T \\
	\hline
	\end{tabular}
	\end{center}
	\item $[p \implies (q \land \neg q)] \iff \neg p$
	\begin{center}
	\begin{tabular}{|c|c|c|c|c|c|c|}
	\hline
	$p$ & $q$ & $\neg q$ & $q \land \neg q$ & $p \implies (q \land \neg q)$ & $[p \implies (q \land \neg q)] \iff \neg p$ & $\neg p$ \\
	\hline
	T & T & F & F & F & T & F\\
	T & F & T & F & F & T & F\\
	F & T & F & F & T & T & T\\
	F & F & T & F & T & T & T\\
	\hline
	\end{tabular}
	\end{center}
\end{enumerate}

\section{Construct a truth table for each statement.}
\begin{enumerate}
	\item $\neg p \lor q$
	\begin{center}
	\begin{tabular}{|c|c|c|c|}
	\hline
	$p$ & $q$ & $\neg p$ & $\neg p \lor q$\\
	\hline
	T & T & F & T\\
	T & F & F & F\\
	F & T & T & T\\
	F & F & T & T\\
	\hline
	\end{tabular}
	\end{center}
	\item $p \land \neg q$
	\begin{center}
	\begin{tabular}{|c|c|c|c|}
	\hline
	$p$ & $q$ & $\neg q$ & $p \land \neg q$\\
	\hline
	T & T & F & F\\
	T & F & T & T\\
	F & T & F & F\\
	F & F & T & F\\
	\hline
	\end{tabular}
	\end{center}
	\item $[\neg q \land (p \implies q)] \implies \neg p$
	\begin{center}
	\begin{tabular}{|c|c|c|c|c|c|c|}
	\hline
	$p$ & $q$ & $\neg q$ & $p \implies q$ & $\neg q \land (p \implies q)$ & $\neg p$ & $[\neg q \land (p \implies q)] \implies \neg p$\\
	\hline
	T & T & F & T & F & F & T\\
	T & F & T & F & F & F & T\\
	F & T & F & T & F & T & T\\
	F & F & T & T & T & F & F\\
	\hline
	\end{tabular}
	\end{center}
\end{enumerate}

\section{Indicate whether each statement is True or False.}
\begin{enumerate}
	\item $2 \leq 3$ and 7 is prime.
	\subitem True
	\item $6+2=8$ or 6 is prime.
	\subitem True
	\item 5 is not prime or 8 is prime.
	\subitem False
	\item If 3 is prime, then $3^2=9$.
	\subitem True
	\item If 3 is not prime, then $3^2 \neq 9$.
	\subitem True
	\item If $3^2 = 9$, then 3 is not prime.
	\subitem False
	\item If 6 is even or 4 is odd, then 6 is prime.
	\subitem False
	\item If $2<3$ implies that $4>5$, then 8 is prime.
	\subitem True
	\item If both $2+5=7$ and $2\cdot5=7$, then $2^2 + 5^2 = 7^2$.
	\subitem True
	\item It is not the case that $2+3\neq5$.
	\subitem True
\end{enumerate}

\section{Indicate whether each statement is True or False.}
\begin{enumerate}
	\item 5 is odd and 3 is even.
	\subitem False
	\item 5 is odd or 3 is even.
	\subitem True
	\item 6 is prime or 8 is odd.
	\subitem False
	\item If 7 is odd, then $7+7=10$.
	\subitem False
	\item If $2+2=5$, then 5 is prime.
	\subitem True
	\item If $4>5$, then 5 is even.
	\subitem True
	\item If 5 is odd and 6 is prime, then $5+6=11$.
	\subitem True
	\item If $5\leq3$ only if 3 is odd, then 5 is even.
	\subitem True
	\item If both $2+5=7$ and $2\cdot5=10$, then $2^2+5^2=10^2$.
	\subitem False
	\item It is not the case that 4 is even and 7 is not prime.
	\subitem True
\end{enumerate}

\end{document}
