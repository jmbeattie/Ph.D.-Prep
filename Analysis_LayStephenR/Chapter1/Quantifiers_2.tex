% !TEX TS-program = pdflatex
% !TEX encoding = UTF-8 Unicode

% This is a simple template for a LaTeX document using the "article" class.
% See "book", "report", "letter" for other types of document.

\documentclass[11pt]{article} % use larger type; default would be 10pt

\usepackage[utf8]{inputenc} % set input encoding (not needed with XeLaTeX)

%%% Examples of Article customizations
% These packages are optional, depending whether you want the features they provide.
% See the LaTeX Companion or other references for full information.

%%% PAGE DIMENSIONS
\usepackage{geometry} % to change the page dimensions
\geometry{a4paper} % or letterpaper (US) or a5paper or....
% \geometry{margin=2in} % for example, change the margins to 2 inches all round
% \geometry{landscape} % set up the page for landscape
%   read geometry.pdf for detailed page layout information

\usepackage{graphicx} % support the \includegraphics command and options

% \usepackage[parfill]{parskip} % Activate to begin paragraphs with an empty line rather than an indent

%%% PACKAGES
\usepackage{booktabs} % for much better looking tables
\usepackage{array} % for better arrays (eg matrices) in maths
\usepackage{paralist} % very flexible & customisable lists (eg. enumerate/itemize, etc.)
\usepackage{verbatim} % adds environment for commenting out blocks of text & for better verbatim
\usepackage{subfig} % make it possible to include more than one captioned figure/table in a single float
\usepackage{enumitem} \setenumerate[0]{label=(\alph*)}
\usepackage{amssymb}
\usepackage{amsmath}
% These packages are all incorporated in the memoir class to one degree or another...

%%% HEADERS & FOOTERS
\usepackage{fancyhdr} % This should be set AFTER setting up the page geometry
\pagestyle{fancy} % options: empty , plain , fancy
\renewcommand{\headrulewidth}{0pt} % customise the layout...
\lhead{}\chead{}\rhead{}
\lfoot{}\cfoot{\thepage}\rfoot{}

%%% SECTION TITLE APPEARANCE
\usepackage{sectsty}
\allsectionsfont{\normalfont\large\sffamily} % (See the fntguide.pdf for font help)
% (This matches ConTeXt defaults)

%%% ToC (table of contents) APPEARANCE
\usepackage[nottoc,notlof,notlot]{tocbibind} % Put the bibliography in the ToC
\usepackage[titles,subfigure]{tocloft} % Alter the style of the Table of Contents
\renewcommand{\cftsecfont}{\rmfamily\mdseries\upshape}
\renewcommand{\cftsecpagefont}{\rmfamily\mdseries\upshape} % No bold!

%%% END Article customizations

%%% The "real" document content comes below...

\title{Quantifiers}
\author{Jon Beattie}
%\date{} % Activate to display a given date or no date (if empty),
         % otherwise the current date is printed 

\begin{document}
\maketitle

\section{Mark each statement as True or False. Justify each answer.}
\begin{enumerate}
\item The symbol ``$\forall$'' means ``for every''.
\subitem True; This is the definition of ``$\forall$''
\item The negation of a universal statement is another universal statement.
\subitem False; The negation of a universal statement is an existential statement.
\item The symbol ``$\ni$'' is read ``such that''.
\subitem True; This is the definition of ``$\ni$''
\end{enumerate}

\section{Mark each statement True or False. Justify each answer.}
\begin{enumerate}
\item The symbol ``$\exists$'' means ``there exist several''.
\subitem False; ``$\exists$'' means ``there exists'' which may only refer to a single entity.
\item If a variable is used in the antecedent of an implication without being quantified, then the universal quantifier is assumed to apply.
\subitem True; This is a general rule.
\item The order in which quantifiers are used affects the truth value.
\subitem True; Suppose I were to say for every number $x$ there exists some number $y$ such that $x-y=0$. This is true for $x=y$. However, If I were to say there exists some $y$ such that for all $x$, $x-y=0$, this would be false. If $x\neq y$, then this assertion breaks.
\end{enumerate}

\section{Write the negation of each statement.}
\begin{enumerate}
\item All the roads in Yellowstone are open.
\subitem Some road in Yellowstone is closed.
\item Some fish are green.
\subitem All fish are not green.
\item No even integer is prime.
\subitem There exists an even integer that is prime.
\item $\exists x<3\ni x^2 \geq 10$.
\subitem $\forall x>3, x^2>10$
\item $\forall x$ in $A, \exists y <k \ni 0 <f(y)<f(x)$
\subitem $\exists x$ in $A \ni \forall y<k,0\geq f(y)\geq f(x)$
\item If $n>N$, then $\forall x$ in $S$, $\mid f_n(x)-f(x)\mid<\epsilon$.
\subitem $n>N$ and $\exists x$ in $S \ni \mid f_n(x)-f(x)\mid\geq\epsilon$.
\end{enumerate}

\section{Write the negation of each statement.}
\begin{enumerate}
\item Some basketball players at Central High are short.
\subitem All basketball players at Central High are tall.
\item All of the lights are on.
\subitem Some of the lights are off.
\item No bounded interval contains infinitely many integers.
\subitem Some bounded interval doesn't contain infinitely many integers.
\item $\exists x$ in $S \ni x \geq 5$.
\subitem $\forall x$ in $S \ni x<5$.
\item $\forall x \ni 0 <x<1, f(x)<2$ or $ f(x)>5$.
\subitem$\exists x \ni 0<x<1, 2<f(x)<5$.
\item If $x>5$, then $\exists y>0\ni x^2>25+y$.
\subitem $x>5$ and $\forall y>0, x^2<25+y$.
\end{enumerate}

\section{Determine the truth value of each statement, assuming that $x,y$, and $z$ are real numbers.}
\begin{enumerate}
\item $\exists x\ni\forall y \exists z\ni x+y=z$.
\subitem True
\item $\exists x \ni\forall y$ and $\forall z, x+y=z$.
\subitem False
\item $\forall x$ and $\forall y,\exists z \ni y-z=x$.
\subitem True
\item $\forall x$ and $\forall y,\exists z \ni xz=y$.
\subitem False
\item $\exists x \ni \forall y$ and $\forall z, z>y$ implies that $z>x+y$.
\subitem True
\item $\forall x, \exists y$ and $\exists z \ni z>y$ implies that $z>x+y$.
\subitem True
\end{enumerate}

\section{Determine the truth value of each statement, assuming that $x,y$ and $z$ are real numbers.}
\begin{enumerate}
\item $\forall x$ and $\forall y, \exists z \ni x+y=z$.
\subitem True
\item $\forall x \exists y \ni \forall z, x+y=z$.
\subitem False
\item $\exists x \ni \forall y, \exists z \ni xz=y$.
\subitem True
\item $\forall x$ and $\forall y, \exists z \ni yz=x$.
\subitem False
\item $\forall x \exists y \ni \forall z, z>y$ implies that $ z>x+y$.
\subitem False
\item $\forall x$ and $\forall y, \exists z \ni z>y$ implies that $z>x+y$.
\subitem True
\end{enumerate}

\section{Below are two strategies for determining the truth value of a statement involving a positive number $x$ and another statement $P(x)$.
Find some $x>0$ such that $P(x)$ is true.
Let $x$ be the name for any number greater than 0 and show $P(x)$ is true.
For each statement below, indicate which strategy is more appropriate.}
\begin{enumerate}
\item $\forall x>0, P(x)$.
\subitem Use the second method.
\item $\exists x>0 \ni P(x)$.
\subitem Use the first method.
\item $\exists x>0 \ni \neg P(x)$.
\subitem Use the first method.
\item $\forall x>0, \neg P(x)$.
\subitem Use the second method.
\end{enumerate}

\section{Which of the following best identifies $f$ as a constant function, where $x$ and $y$ are real numbers}
\begin{enumerate}
\item $\exists x \ni \forall y, f(x)=y$.
\subitem This isn't a function.
\item $\forall x \exists y \ni f(x) = y$.
\subitem This isn't necessarily constant.
\item $\exists y \ni \forall x, f(x) = y$.
\subitem $f$ is constant.
\item $\forall y \exists x \ni f(x) = y$.
\subitem This isn't constant.
\end{enumerate}

\section{Determine the truth value of each statement, assuming $x$ is a real number.}
\begin{enumerate}
\item $\exists x$ in $\left[ 2,4 \right] \ni x<7$.
\item $\forall x$ in $\left[ 2,4 \right], x<7$.
\item $\exists x \ni x^2=5$.
\item $\forall x, x^2=5$.
\item $\exists x \ni x^2 \neq -3$.
\item $\forall x, x^2 \neq -3$.
\item $\exists x \ni x \div x = 1$.
\item $\forall x, x \div x = 1$.
\end{enumerate}

\section{Determine the truth value of each statement, assuming $x$ is a real number.}
\begin{enumerate}
\item $\exists x$ in $\left[ 3,5 \right]\ni x\geq4$.
\item $\forall x$ in $\left[ 3,5 \right], x\geq4$.
\item $\exists x \ni x^2 \neq 3$.
\item $\forall x, x^2 \neq 3$.
\item $\exists x \ni x^2 = -5$.
\item $\forall x, x^2 = -5$.
\item $\exists x \ni x-x=0$.
\item $\forall x, x-x=0$.
\end{enumerate}


\end{document}
